% This file was converted to LaTeX by Writer2LaTeX ver. 1.4
% see http://writer2latex.sourceforge.net for more info
\documentclass{article}
\usepackage[utf8]{inputenc}
\usepackage[T1]{fontenc}
\usepackage[english,polish]{babel}
\let\lll\undefined
\usepackage{amsmath}
\usepackage{amssymb,amsfonts,textcomp}
\usepackage{color}
\usepackage{array}
\usepackage{supertabular}
\usepackage{hhline}
\usepackage{hyperref}
\hypersetup{pdftex, colorlinks=true, linkcolor=blue, citecolor=blue, filecolor=blue, urlcolor=blue, pdftitle=, pdfauthor=, pdfsubject=, pdfkeywords=}
\usepackage[pdftex]{graphicx}
\makeatletter
\newcommand\arraybslash{\let\\\@arraycr}
\makeatother

% Page layout (geometry)
\setlength\paperwidth{21.001cm}
\setlength\paperheight{30.582cm}
\setlength\voffset{-1in}
\setlength\hoffset{-1in}
\setlength\topmargin{3.256cm}
\setlength\oddsidemargin{2.163cm}
\setlength\textheight{23.675999cm}
\setlength\textwidth{16.892cm}
\setlength\footskip{2.401cm}
\setlength\headheight{0cm}
\setlength\headsep{0cm}

% Footnote rule
\setlength{\skip\footins}{0.119cm}
\renewcommand\footnoterule{\vspace*{-0.018cm}\setlength\leftskip{0pt}\setlength\rightskip{0pt plus 1fil}\noindent\textcolor{black}{\rule{0.0\columnwidth}{0.018cm}}\vspace*{0.101cm}}

% Pages styles
\makeatletter
\newcommand\ps@Standard{
  \renewcommand\@oddhead{}
  \renewcommand\@evenhead{}
  \renewcommand\@oddfoot{\thepage{}}
  \renewcommand\@evenfoot{\@oddfoot}
  \renewcommand\thepage{\arabic{page}}
}
\makeatother
\pagestyle{Standard}
\setlength\tabcolsep{1mm}
\renewcommand\arraystretch{1.3}

\usepackage{multirow}
\usepackage{listingsutf8}
\usepackage{csvsimple}


% Default fixed font does not support bold face
\DeclareFixedFont{\ttb}{T1}{txtt}{bx}{n}{12} % for bold
\DeclareFixedFont{\ttm}{T1}{txtt}{m}{n}{12}  % for normal

% Custom colors
\usepackage{color}
\definecolor{deepblue}{rgb}{0,0,0.5}
\definecolor{deepred}{rgb}{0.6,0,0}
\definecolor{deepgreen}{rgb}{0,0.5,0}

\usepackage{listings}
\usepackage{epstopdf}



% Python style for highlighting
\newcommand\pythonstyle{\lstset{
language=Python,
basicstyle=\ttm,
otherkeywords={self},             % Add keywords here
keywordstyle=\ttb\color{deepblue},
emph={MyClass,__init__},          % Custom highlighting
emphstyle=\ttb\color{deepred},    % Custom highlighting style
stringstyle=\color{deepgreen},
frame=tb,                         % Any extra options here
showstringspaces=false            % 
inputencoding=utf8,
literate=%
{ą}{{\k{a}}}1
{Ą}{{\k{A}}}1
{ć}{{\'c}}1
{Ć}{{\'{C}}}1
{ę}{{\k{e}}}1
{Ę}{{\k{E}}}1
{ł}{{\l{}}}1
{Ł}{{\L{}}}1
{ń}{{\'n}}1
{Ń}{{\'N}}1
{ó}{{\'o}}1
{Ó}{{\'O}}1
{ś}{{\'s}}1
{Ś}{{\'S}}1
{ż}{{\.z}}1
{Ż}{{\.Z}}1
{ź}{{\'z}}1
{Ź}{{\'Z}}1
}}
\usepackage{wrapfig}


\title{Sprawozdanie: Porównanie Algorytmów Sortowania}
\author{Krzysztof Stasiowski}
\date{2018-01-07}
\begin{document}

\clearpage
\setcounter{page}{1}
\begin{flushleft}
\tablefirsthead{}
\tablehead{}
\tabletail{}
\tablelasttail{}
\begin{supertabular}{|m{2.552cm}|m{4.4300003cm}m{4.4300003cm}m{4.3240004cm}|}
\hline
\multirow{3}{*}{|m{2.552cm}|}{
	\centering
 \includegraphics[width=1.983cm,height=3.868cm]{AGH-logo.jpg} 

} &
\multicolumn{3}{m{13.584001cm}|}{\centering{\bfseries Akademia Górniczo-Hutnicza im. Stanisława Staszica w Krakowie
}}\\\hhline{~---}
 &
\multicolumn{3}{m{13.584001cm}|}{\centering{\bfseries Wstęp do Informatyki}}\\\hhline{~---}
 &
\multicolumn{1}{m{4.4300003cm}|}{{\itshape Wydział{\l}:}
\centering{\bfseries EAIiIB}} &
\multicolumn{1}{m{4.4300003cm}|}{{\itshape Kierunek:}

\centering \textbf{\ \ \ Informatyka}} &
{\itshape Imi\k{e} i nazwisko:}
\centering\arraybslash{\bfseries Krzysztof Stasiowski}\\\hhline{~--~}
 &
\multicolumn{1}{m{4.4300003cm}|}{{\itshape Rok i semestr:}

~

\centering{\bfseries 2017/2018, I}} &
\multicolumn{1}{m{4.4300003cm}|}{{\itshape Data \'cwiczenia:}

~

\centering{\bfseries  2018-01-07}} &
\\\hhline{----}
\end{supertabular}
\end{flushleft}

%stupidity shuld have ended
\section{Cel Ćwiczenia}
Celem Ćwiczenia Zajęcia z UNIXa w dn. 18-19.12 odbywają się w trybie zdalnym samodzielnie/w parach należało zrealizować instrukcję do zajęć z programowania systemowego. Dodatkowo należało wykonać następujące polecenia:
\begin{enumerate}
\item Proszę przesłać rozbudowany program f1.c.
\item Proszę przesłać rozbudowany program d2.c.
\item Jaka jest różnica pomiędzy funkcjami system() a exec() (p2.c)?
\item Jak działa program p3.c? Dlaczego?
\item Proszę otoczyć komentarzem wywołanie funkcji sleep() w p4.c, jak to wpłynie na działanie procesów? Proszę wytłumaczyć.\ldots 
\end{enumerate}

\section{Rozbudowany Program}







\end{document}
