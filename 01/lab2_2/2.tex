\documentclass[a4paper,12pt]{article}
\usepackage{amssymb}
\usepackage{amsthm}
\usepackage[T1]{fontenc}
\usepackage[utf8]{inputenc}
\usepackage{times}
\usepackage{anysize}
\usepackage[polish]{babel}
\usepackage{array}
\marginsize{1.5cm}{1.5cm}{1.5cm}{1.5cm}
\sloppy 

\theoremstyle{definition}
\newtheorem{df}{Definicja}


\begin{document}

% \maketitle

\section{Zbiory rozmyte}

Pojęcie zbioru rozmytego zostało wprowadzone przez L. A. Zadeha w 1965. Stosowanie zbiorów rozmytych w systemach sterownia pozwala na dokładniejsze odwzorowanie pojęć stosowanych przez ludzi, które często są subiektywne i nieprecyzyjne. Stopniowe przejście między przynależnością do zbioru a jej brakiem pozwala nam uniknąć ścisłej klasyfikacji elementów, która często jest niemożliwa. Logika rozmyta jest uogólnieniem logiki klasycznej. 

Koncepcja zbiorów rozmytych wyrosła na gruncie logicznego formalizowania pomysłu, aby elementom zbioru przypisywać tzw. stopień przynależności do zbioru, określający wartość prawdziwości danego wyrażenia za pomocą liczb z przedziału \(\left[0,1\right]\). 

\begin{df}
\textbf{Zbiorem rozmytym} $A$ w pewnej niepustej przestrzeni $X$ nazywamy zbiór uporządkowanych par:
\begin{equation}
A = \left\{
\left( 
x , \mu_A(x)
\right)\!\colon x \in X
\right\}
\end{equation}

gdzie:
\begin{equation}
\mu_A \colon X \rightarrow \left[0,1\right]
\end{equation} 
jest \textbf{funkcją przynależności} zbioru \(A\).
Wartość funkcji \(\mu_A\) w punkcie $x$ \textbf{stopniem przynależności} $x$ do zbioru \(A\). 
\end{df}


Zbiory rozmyte $A$, \(B\), \(C\) itd. względem \(X\) nazywamy również \textbf{podzbiorami rozmytymi} w \(X\).
Zbiór wszystkich zbiorów rozmytych w \(X\) oznaczamy przez \(F(X)\). Zbiory klasyczne można interpretować jako zbiory rozmyte z~funkcją przynależności przyjmującą tylko wartości 0 i 1.

\medskip
\noindent
{\bf Przykład:} Niech  $A$ oznacza zbiór niskich temperatur i niech \(X=[-5,50]\). Funkcja przynależności może być określona następująco:
\begin{equation}
\mu_A=\left\lbrace
\begin{array}[c]{ll}
1 					&:x<10\\
- \frac{1}{10}+2 	&:10\leqslant x\leqslant20 \\
0					&:x>10
\end{array}
\right.%można było użyć cases
\end{equation}

W przypadku, gdy ustalony zbiór $X$ jest skończony, tzn. \(X=\{x_1,x_2,x_3,\dots,x_n\}\), funkcje przynależności zbiorów rozmytych można przedstawiać za pomocą tabelek, stosować zapis w postaci sumy:
\begin{equation}
A=\sum_{i=1}^n{\frac{\mu_A(x)}{x}}
\end{equation}
(kreski ułamkowe i znaki sumy należy rozumieć czysto symbolicznie) lub podawać elementy zbioru w postaci par:
\begin{equation}
A=\{
(x_1,\mu_A(x_2)),
(x_2,\mu_A(x_1)),
\dots,
(x_n,\mu_A(x_n))
\}
\end{equation}

\end{document}

