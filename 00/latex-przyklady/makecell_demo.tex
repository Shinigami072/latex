\documentclass[a4paper,11pt]{article}
\usepackage[polish]{babel}
\usepackage[utf8]{inputenc}
\usepackage[T1]{fontenc}
\usepackage{times}
\usepackage{graphicx}
\usepackage{anysize}
\usepackage{makecell}
\usepackage{rotating}
\usepackage{multirow}

%\marginsize{left}{right}{top}{bottom}
\marginsize{2.5cm}{2.5cm}{2.5cm}{2.5cm}


\begin{document}


\begin{tabular}{|l|l|l|}
\hline
Rodzina & Gatunek & Wychów piskląt \\ \hline
Rodzina krukowatych & Kruk & Pisklęta są gniazdownikami \\ \hline
Rodzina krukowatych & Gawron & Pisklęta są gniazdownikami \\ \hline
Rodzina jaskółkowatych & Jaskółka dymówka & Pisklęta są rzekomymi gniazdownikami \\ \hline
Rodzina jaskółkowatych & Jaskółka oknówka & Pisklęta są rzekomymi gniazdownikami \\ \hline
%
Rodzina jaskółkowatych & Jaskółka oknówka & 
\makecell[l]{Pisklęta są rzekomymi\\ gniazdownikami} \\ \hline
%
Rodzina jaskółkowatych & Jaskółka oknówka & 
\makecell[c]{Pisklęta są rzekomymi\\ gniazdownikami} \\ \hline
%
Rodzina jaskółkowatych & Jaskółka oknówka & 
\makecell[r]{Pisklęta są rzekomymi\\ gniazdownikami} \\ \hline
%
Rodzina jaskółkowatych & Jaskółka oknówka & 
\makecell*[l]{Pisklęta są rzekomymi\\ gniazdownikami} \\ \hline
%
Rodzina jaskółkowatych & Jaskółka oknówka & 
\makecell*[c]{Pisklęta są rzekomymi\\ gniazdownikami} \\ \hline
%
Rodzina jaskółkowatych & Jaskółka oknówka & 
\makecell*[r]{Pisklęta są rzekomymi\\ gniazdownikami} \\ \hline
%
Rodzina jaskółkowatych & Jaskółka oknówka & 
\makecell[tl]{Pisklęta są rzekomymi\\ gniazdownikami} \\ \hline
%
\makecell[bl]{Rodzina\\ jaskółkowatych} & 
Jaskółka oknówka & 
\makecell[cl]{Pisklęta są rzekomymi\\ gniazdownikami} \\ \hline
%
Rodzina jaskółkowatych & Jaskółka oknówka & 
\makecell[bl]{Pisklęta są rzekomymi\\ gniazdownikami} \\ \hline
%
Rodzina jaskółkowatych & Jaskółka oknówka & 
\makecell[{}{p{2.5cm}}]{Pisklęta są rzekomymi gniazdownikami} \\ \hline
\end{tabular}

\vspace{1cm}

\begin{tabular}{|l|l|l|}
\hline
\thead{Rodzina} & \thead{Gatunek} & \thead{Wychów\\ piskląt} \\ \hline
Rodzina krukowatych & Kruk & Pisklęta są gniazdownikami \\ \hline
Rodzina krukowatych & Gawron & Pisklęta są gniazdownikami \\ \hline
\end{tabular}

\vspace{1cm}

\renewcommand\theadalign{tl}
\renewcommand\theadfont{\bfseries\normalsize}

\begin{tabular}{|l|l|l|}
\hline
\thead{Rodzina} & \thead{Gatunek} & \thead{Wychów\\ piskląt} \\ \hline
Rodzina krukowatych & Kruk & Pisklęta są gniazdownikami \\ \hline
Rodzina krukowatych & Gawron & Pisklęta są gniazdownikami \\ \hline
\end{tabular}

\vspace{1cm}


\setlength\rotheadsize{15mm}

\begin{tabular}{|l|l|l|}
\hline
\rothead{Rodzina} & \rothead{Gatunek} & \rothead{Wychów\\ piskląt} \\ \hline
Rodzina krukowatych & Kruk & Pisklęta są gniazdownikami \\ \hline
Rodzina krukowatych & Gawron & Pisklęta są gniazdownikami \\ \hline
\end{tabular}

\vspace{1cm}

\begin{tabular}{|l|l|}
\hline
Rodzina & Gatunek \\ \hline
Rodzina krukowatych & Kruk \\ \hline
\Gape[2mm]{Rodzina krukowatych} & Kruk \\ \hline
\Gape[4mm][0mm]{Rodzina krukowatych} & Kruk \\ \hline
\Gape[1mm][4mm]{Rodzina krukowatych} & Kruk \\ \hline
\end{tabular}


\vspace{1cm}

\setcellgapes{15pt}

\begin{table}[!ht]
\makegapedcells
\begin{tabular}{|l|l|l|}
\hline
Rodzina & Gatunek & Wychów piskląt \\ \hline
Rodzina krukowatych & Kruk & Pisklęta są gniazdownikami \\ \hline
Rodzina krukowatych & Gawron & Pisklęta są gniazdownikami \\ \hline
\end{tabular}
\end{table}

\begin{table}[!ht]
\begin{tabular}{|l|l|l|}
\hline
Rodzina & Gatunek & Wychów piskląt \\ \hline
Rodzina krukowatych & Kruk & Pisklęta są gniazdownikami \\ \hline
Rodzina krukowatych & Gawron & Pisklęta są gniazdownikami \\ \hline
\end{tabular}
\end{table}

\vspace{1cm}

\begin{tabular}{|l|l|l|}
\hline
Rodzina & Gatunek & Wychów piskląt \\ \hline
\multirowcell{2}[0pt][l]{Rodzina krukowatych} & 
Kruk & Pisklęta są gniazdownikami \\ \cline{2-3}
 & Gawron & Pisklęta są gniazdownikami \\ \hline
\multirowcell{2}[0pt][l]{Rodzina jaskółkowatych} & 
Jaskółka dymówka & Pisklęta są rzekomymi gniazdownikami \\ \cline{2-3}
 & Jaskółka oknówka & Pisklęta są rzekomymi gniazdownikami \\ \hline
\end{tabular}


\vspace{1cm}

\renewcommand\theadalign{cc}
\renewcommand\theadfont{\normalsize}

\begin{tabular}{|l|l|l|}
\hline
\diaghead(2,-1){xxxxxxxxxxxxxxxxxx}  % tekst na podstawie którego definiowana jest szerokość kolumny
{Krukowate}{Jaskółkowate} &
\thead{Jaskółka dymówka}&
\thead{Jaskółka oknówka}  \\ \hline
Kruk && \\\hline
Gawron && \\\hline
\end{tabular}




\end{document}
