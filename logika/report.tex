\documentclass{report-agh}
\title{Zadanie domowe II}
\author{Krzysztof Stasiowski\\Weronika Niedzwiedź\\Daniel Gut\\Michał Borkowski\\Przemysław Bielecki}
\ASemester{2}
\AYear{2017/2018}
\Subject{Logika}
%\let\nline\udefined
\usepackage{fitch}
\usepackage{enumitem}
\usepackage{prooftrees}

\begin{document}

\maketitle

\section{Cel Ćwiczenia}
\begin{enumerate}[label=zad \arabic*.]
\item
Korzystając z metody tablic semantycznych sprawdzić czy następująca formuła jest formułą spełnialną lub tautologią: 
$(( r \land  \lnot q) \land ( \lnot r \lor \lnot q)) \implies (\lnot q \implies (p \land r))$
\item
Za pomocą systemu Fitch udowodnić konkluzję $ (\lnot p \implies \lnot q)\implies p$\\
Z przesłanki  $\lnot p \implies q$
\end{enumerate}

\newcommand{\BimplElim}{ $\beta$ E $\implies$}
\newcommand{\BandElim}{ $\beta$ E $\lnot(\land)$}
\newcommand{\BorElim}{ $\beta$ E $\lor$}
\newcommand{\BiffElim}{ $\beta$ E $\lnot(\iff)$}

\newcommand{\AnotElim}{ $\alpha$ E $\lnot\lnot$}
\newcommand{\AimplElim}{ $\alpha$ E $\lnot(\implies)$}
\newcommand{\AandElim}{ $\alpha$ E $\land$}
\newcommand{\AorElim}{ $\alpha$ E $\lnot(\lor)$}
\newcommand{\AiffElim}{ $\alpha$ E $\iff$}

\section{zad 1. Tablice Semantyczne}
\begin{prooftree}
  {
  	to prove=(( r \land  \lnot q) \land ( \lnot r \lor \lnot q)) \implies (\lnot q \implies (p \land r))
    %close with={\ensuremath{\ast}}
  }
  [(( r \land  \lnot q) \land ( \lnot r \lor \lnot q)) \implies (\lnot q \implies (p \land r)
	[\lnot (( r \land  \lnot q) \land ( \lnot r \lor \lnot q)), just={\BimplElim}
		[\lnot(r \land \lnot q), just={\BandElim}
			[\lnot r, just={\BandElim}]
			[\lnot\lnot q, just={\BandElim}
				[q, just={\AnotElim}]			
			]
		]
		[\lnot(\lnot r \lor \lnot q), just={\BandElim}
			[\lnot\lnot r.\lnot\lnot q), just={\AorElim}
				[ r.q, just={\AnotElim}]
			]
		]
  	]
  	[\lnot q \implies (p \land r), just={\BimplElim}
  		[\lnot\lnot q , just={\BimplElim}
  			[q, just={\AnotElim}]
  		]
  		[	p \land r
  			[p.r,just={\AandElim}]
  		]
  	]
  ]
\end{prooftree}

\section{zad 2. Fitch}
\begin{fitch}
	\fj \lnot p \implies q 					&Przesłanka\\%1
	\fa \fh \lnot p \implies \lnot q			&Założenie\\%2
	\fa \fa \lnot p \implies q 				&Reiteracja(1)\\%3
	\fa \fa p 								&Wprowadzenie $\lnot$(2,3)\\%4
	(\lnot p \implies \lnot q )\implies p    &Wprowadzenie $\implies$ (4)%5
\end{fitch}
\end{document}