\documentclass[11pt,a4paper,final]{article}
\usepackage{amsmath}
\usepackage{amsfonts}
\usepackage{amssymb}
\usepackage[utf8]{inputenc}
\usepackage[polish]{babel}
\usepackage[T1]{fontenc}
\usepackage{array}
\usepackage{anysize}
\marginsize{2cm}{2cm}{2cm}{2cm}
\author{Krzysztof Stasiowski}

\def\arraystretch{1.2}%odstęp między wierszami


\begin{document}
Istnieje ścisły związek między rozkładem macierzy $A$ na macierze $L$ i $U$
a metodą eliminacji Gaussa.
Można wykazać,że elementy kolejnych kolumn macierzy $L$ są równe współczynnikom przez które mnożone są w kolejnych krokach wiersze układu równań celem dokonania eliminacji niewiadomych w odpowiednich kolumnach. Natomiast macierz $U$ jest równa macierzy trójkątnej uzyskanej w eliminacji Gaussa.
\begin{equation*}
[A|b]=
\left[\begin{array}[c]{rrrr}
2&2&4&4\\
1&2&2&4\\
1&4&1&1
\end{array}\right]
=
\left[\begin{array}[c]{rrrr}
2&2&4&4\\
0&1&0&2\\
0&3&-1&-1
\end{array}\right]
=
\left[\begin{array}[c]{rrrr}
2&2&4&4\\
0&1&0&2\\
0&0&-1&-7
\end{array}\right]
\end{equation*}
\begin{equation*}
L=
\left[\begin{array}[c]{rrr}
1&0&0\\
\frac{1}{2}&1&0\\
\frac{1}{2}&3&1
\end{array}\right]
%
\qquad
U=
\left[\begin{array}[c]{rrrr}
2&2&4&4\\
0&1&0&2\\
0&0&-1&-7
\end{array}\right]
\end{equation*}

\end{document}